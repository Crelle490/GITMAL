COURSE
DEFS 
	[HOMEHTML] https://itundervisning.ase.au.dk/ITMAL_E21/Html
	[HOME]     https://itundervisning.ase.au.dk/ITMAL_E21
	[FIGS]     https://itundervisning.ase.au.dk/ITMAL_E21/Html/Figs
	[HOML]     <span style='font-family: courier new, courier;'>[HOML]</span>
	[GITMAL]   <span style='font-family: courier new, courier;'>[GITMAL]</span>
	[GITHOML]  <span style='font-family: courier new, courier;'>[GITHOML]</span>
	[JPYNB]    <span style='font-family: courier new, courier;'>[JPYNB]</span>
	[OPTIONAL] (OPTIONEL)

	[KURSUSINFORMATION]  <a href='https://brightspace.au.dk/d2l/le/lessons/27524/units/244588'>kursusinformation</a>
	[KURSUSFORKORTELSER] <a href='https://brightspace.au.dk/d2l/le/lessons/27524/topics/254943' rel='noopener' target='_blank'>kursusinformation | kursusforkortelser</a>
	[KURSUSINFOGPU]      <a href='https://brightspace.au.dk/d2l/le/lessons/27524/topics/244596' rel='noopener' target='_blank'>kursusinformation | GPU Cluster</a>
	
	[BR] <br>

%%%%%%%%%%%%%%%%%%%%%%%%%%%%%%%%%%%%%%%%%%%%%%%%%%%%%%%%%%%%%%%%%%%%%%%%%%%%%%%
%%%%%%%%%%%%%%%%%%%%%%%%%%%%%%%%%%%%%%%%%%%%%%%%%%%%%%%%%%%%%%%%%%%%%%%%%%%%%%%

CONTENT Litteratur

%%%%%%%%%%%%%%%%%%%%%%%%%%%%%%%%%%%%%%%%%%%%%%%%%%%%%%%%%%%%%%%%%%%%%%%%%%%%%%%
%%%%%%%%%%%%%%%%%%%%%%%%%%%%%%%%%%%%%%%%%%%%%%%%%%%%%%%%%%%%%%%%%%%%%%%%%%%%%%%

%\header{Litteratur}

\sub{Hands-on Machine Learning [HOML]}

\displaystyle{
	\img{[FIGS]/book_homl.jpg, Hands-on Machine Learning with Scikit-Learn (front image)}

	[BR] \i{Hands-on Machine Learning with Scikit-Learn, Keras, and TensorFlow: Concepts, Tools, and Techniques to Build Intelligent Systems}
	[BR]
	
	[BR] Aurélien Géron 
	[BR] O'Reilly / Wiley, 2019, 2.ed.
	[BR] ISBN: 9781492032649 
	[BR] \link{https://www.oreilly.com/library/view/hands-on-machine-learning/9781492032632/}
}

\p{\i{NOTE 1:} dette er anden udgave (Second Edition/2.ed) af Géron's " Hands-on",
undgå at bruge førsteudgaven, idet den benytter TensorFlow direkte istedet for
Keras, og desuden har flere mangler.}

\p{\i{NOTE 2:} i PDF udgaven (Early Release, June 2019, 2019-04-22: Fifth Release)
svare sidetal og nogle kaptitler ikke til den officielle bog udgave ovenfor!}

\sub{Deep Learning [DL]}

\displaystyle{

	\img{[FIGS]/book_dl.jpg, Deep Learning (front image)}

	[BR] \i{Deep Learning}
	[BR]
	
	[BR] Ian Goodfellow, Yoshua Bengio, Aaron Courville
	[BR] The MIT Press 
	[BR] November 18, 2016
	[BR] Hardcover: 775 pages
	[BR] ISBN-10: 0262035618 
	[BR] ISBN-13: 978-0262035613
	[BR] \link{http://www.deeplearningbook.org/}
}

\p{\i{NOTE:} ikke obligatorisk, kun få afsnit og figure bruges herfra.  (Bog god til
videregående Neural Netværks-teori og meget brugt i ML sammenhænge.)}

%%%%%%%%%%%%%%%%%%%%%%%%%%%%%%%%%%%%%%%%%%%%%%%%%%%%%%%%%%%%%%%%%%%%%%%%%%%%%%%
%%%%%%%%%%%%%%%%%%%%%%%%%%%%%%%%%%%%%%%%%%%%%%%%%%%%%%%%%%%%%%%%%%%%%%%%%%%%%%%

CONTENT Kursusforkortelser

%%%%%%%%%%%%%%%%%%%%%%%%%%%%%%%%%%%%%%%%%%%%%%%%%%%%%%%%%%%%%%%%%%%%%%%%%%%%%%%
%%%%%%%%%%%%%%%%%%%%%%%%%%%%%%%%%%%%%%%%%%%%%%%%%%%%%%%%%%%%%%%%%%%%%%%%%%%%%%%

%\header{Kursusforkortelser}

\dl{
	\dt{[DL]:}
	\dd{Enten bare Deep Learning eller Deep Learning bogen af Ian Goodfellow, et. al.}
				
	\dt{[G]:}
	\dd{Group, ITMAL øvelsesgruppe.}
		
	\dt{[GITHOML]:}
	\dd{
		\dl{
			\dt{GitHub repository til [HOML],}
			\dd{\link{https:///github.com/ageron/handson-ml2/}}
			\dt{Clone via HTTPS}
			\dd{\code{git clone https://github.com/ageron/handson-ml2.git}}
			\dt{eller via SSH}
			\dd{\code{git clone git@github.com:ageron/handson-ml2.git}}
		}
	}

	\dt{[GITMAL]:}
	\dd{
		\dl{
			\dt{Git repository for ITMAL,}
			\dd{\link{https://gitlab.au.dk/au204573/GITMAL/}}
			\dt{Clone via HTTPS}
			\dd{\code{git clone https://gitlab.au.dk/au204573/GITMAL.git}}
			\dt{eller via SSH}
			\dd{\code{git clone git@gitlab.au.dk:au204573/GITMAL.git}}
		}
	}

	\dt{[HOML]:}
	\dd{Hands-on Machine Learning af Aurélien Géron, hovedlitteratur til dette kursus. For klarhedens skyld undtales 'HOML' som Holm i Brian Holm.}
	\dd{
		[BR]\img{[FIGS]/brian_holm.jpg, Brian Holm (foto fra cdn-ctstaging.pressidium.com)}
		[BR]\cite{https://cdn-ctstaging.pressidium.com/wp-content/uploads/2020/12/CORVOS_00000365-066.jpg}
	}


	\dt{[ITMAL]:}
	\dd{IT Machine Learning, kursusnavnet.}

	% [J1, J2, .. JN]: En journal opgave/aflevering, f.eks. "Journal 1" (J1). 'Journaler' erstattes af 'opgave afleveringer', O1, O2, osv.

	\dt{[JPYNB]:}
	\dd{Jypyter Python NoteBook, dvs. Notebook applikationen eller en notebook kildetekst fil (med endelsen .ipynb).}

	\dt{[ML]:}
	\dd{Machine Learning, det generelle koncept.}
	
	\dt{[NN:]}
	\dd{Neural Network(s).}

	\dt{[O1, O2, O3, O4]:}
	\dd{En opgaveaflevering, f.eks. O1 for opgaveaflevering 1.} %(opgave afleveringer hed tidligere journaler).

	\dt{[SG]:}
	\dd{Super-group, bestående af tre eller fire Grupper [G]'s.}

	\dt{[Q]:}
	\dd{Et specifikt spørgsmål (Question) i en journal opgave, ala Qc for opgave 'c' i et journal spørgsmål.}
}

%%%%%%%%%%%%%%%%%%%%%%%%%%%%%%%%%%%%%%%%%%%%%%%%%%%%%%%%%%%%%%%%%%%%%%%%%%%%%%%
%%%%%%%%%%%%%%%%%%%%%%%%%%%%%%%%%%%%%%%%%%%%%%%%%%%%%%%%%%%%%%%%%%%%%%%%%%%%%%%

CONTENT Dokumentation og links

%%%%%%%%%%%%%%%%%%%%%%%%%%%%%%%%%%%%%%%%%%%%%%%%%%%%%%%%%%%%%%%%%%%%%%%%%%%%%%%
%%%%%%%%%%%%%%%%%%%%%%%%%%%%%%%%%%%%%%%%%%%%%%%%%%%%%%%%%%%%%%%%%%%%%%%%%%%%%%%

\sub{Web sites}

Primære

\dl{
    \dt{[GITHOML]:}
	    \dd{\link{https://github.com/ageron/handson-ml2/}}
    \dt{Scikit-learn:} 
    	\dd{\link{https://scikit-learn.org/stable/}}
    \dt{Keras:}
    	\dd{\link{https://keras.io/}}
}

Sekundære

\ul{
    \li{\linkex{Jupyter: jupyter-notebook.readthedocs.io/en/stable,  https://jupyter-notebook.readthedocs.io/en/stable/}}
    \li{\linkex{Anaconda Cloud: anaconda.org,                        https://anaconda.org/}}
   	\li{\linkex{Tensorflow: www.tensorflow.org,                      https://www.tensorflow.org/}}
}

Datakilder

\ul{
    \li{\link{Kaggle datasets: www.kaggle.com, https://www.kaggle.com/}}
    \li{\i{Sign in with your email} => genbrug gerne min konto, og undgå tidsplid:
    	\ul{
           \li{user: cef@ase.au.dk}
           \li{password: test123}
        }
	}
}


Dokumentation

\ul{
    \li{Brug den inbyggede hjælp i [JPYNP]}
    \img{[FIGS]/Screenshot_jupyter_help.png,}
}

Guides etc.

\ul{
    \li{\linkex{A Quick Python intro (short), https://www.w3schools.com/python/python_intro.asp}}
    \li{\linkex{A Python tutorial (not so short!), https://docs.python.org/3/tutorial/}}
    \li{Jupyter shortcuts quick guide XXX}
    \li{Scikit-learn reference XXX}
    \li{Scikit-learn cheat sheet XXX}
}

%%%%%%%%%%%%%%%%%%%%%%%%%%%%%%%%%%%%%%%%%%%%%%%%%%%%%%%%%%%%%%%%%%%%%%%%%%%%%%%
%%%%%%%%%%%%%%%%%%%%%%%%%%%%%%%%%%%%%%%%%%%%%%%%%%%%%%%%%%%%%%%%%%%%%%%%%%%%%%%

CONTENT GPU Cluster

%%%%%%%%%%%%%%%%%%%%%%%%%%%%%%%%%%%%%%%%%%%%%%%%%%%%%%%%%%%%%%%%%%%%%%%%%%%%%%%
%%%%%%%%%%%%%%%%%%%%%%%%%%%%%%%%%%%%%%%%%%%%%%%%%%%%%%%%%%%%%%%%%%%%%%%%%%%%%%%

Der er adgang til en GPU baseret server ifbm kurset. Serveren består af en 'master' som kan tilgås via

\dl{
	\dd{\link{http://gpucluster.st.lab.au.dk/}}
}

og fem 'slave' noder med GPU'er via

\dl{	
	\dd{\link{http://gpucluster.st.lab.au.dk/jhub1}}
	\dd{\link{http://gpucluster.st.lab.au.dk/jhub2}}
	\dd{\link{http://gpucluster.st.lab.au.dk/jhub3}}
	\dd{\link{http://gpucluster.st.lab.au.dk/jhub4}}
	\dd{\link{http://gpucluster.st.lab.au.dk/jhub4}}
	\dd{\link{http://gpucluster.st.lab.au.dk/jhub5} (GPU 3090 og ny 4GHz CPU, 3090 har problemer med Tensorflow)}
}

som frit kan benyttes.

\p{Adgang kræver at i er på EDUROAM eller VPN/au access.}

\p{Der er pt. ingen load-balancing på de fire noder så se på gpucluster hjemmesiden, hvilke noder der er mindst belastede (via TOP
dataen).}

\sub{Brug}

\p{Alle grupper har fået deres egen konto, og f.eks. så logger Grp 09 ind som:}

\dl{
	\dd{Login: f21mal09}
	\dd{Password: f21mal09_123}
}

\p{Dvs. brugernavn/login f21malXX hvor XX er jeres ITMAL gruppe og password sammen som brugernavn med med _123
tilføjet.}

\sub{Quick Guide}

\dl{
	\dt{GIT via Jupyter:}
		\dd{I kan clone git repositoret via en '!'-shell commando i Jupyter notepad'en}
		\dd{\code{! git clone https://cfrigaard@bitbucket.org/cfrigaard/itmal}}
		\dd{og så herefter pull'e via}
		\dd{\code{! cd itmal && git pull}}

	\dt{PYTHONPATH}
		\dd{NOTE: PYTHONPATH er IKKE sat (som vi gjorde i L03/modules_and _classes.ipynb), men kan simuleres via}
		\dd{\code{import sys,os sys.path.append(os.path.expanduser('~/itmal'))}}
}

\p{Hvis du kloner GITMAL til itmal som overnfor er der nu automatisk sat en path op til libitaml, prøv det!}

\sub{GPU Hukommelse}

\p{Ved brug af Keras+GPU allokeres automatisk al GPU hukommelse.  Når vi er flere brugere skal i derfor indsætte
følgende i starten af jeres Keras/Tensorflow Jupyternotebook kode:}

	\displaycode{
		import tensorflow as tf
		from keras.backend.tensorflow_backend import set_session
		config = tf.ConfigProto()
		config.gpu_options.per_process_gpu_memory_fraction = 0.05
		config.gpu_options.allow_growth=True
		set_session(tf.Session(config=config))
	}

eller blot

	\displaycode{
		from libitmal import kernelfuns as itmalkernelfuns
		itmalkernelfuns.EnableGPU()
	}

\p{så allokeres kun en brøkdel af GPU hukommelsen! Det ser ud til at growth=True ikke virker, så sæt per_process_gpu_memory_fraction op hvis i har
behov.}

\p{Der kører nu et automatisk startup-script når i logger ind/åbner en nodebook. Se}

	\displaycode{/home/shared/(??)/startup/00_init.py}

\p{der kører StartupSequence_SetPath() og StartupSequence_EnableGPU(), den sidste unktion med følgende default
paramete}

	\displaycode{def StartupSequence_EnableGPU(gpu_mem_fraction=0.05, gpus=None, cpus=None)}

\p{Bemærk at jeres jupyter server, beholder all hukommelse, også når i logger af...kun "stop my server"/"start server"
frigiver!}

%\p{Jeg vil slå alle proceser ned, der har allokeret over ca.  4Gb GPU hukommelse eller har kørt i en uge...det er en
%automatisk process, der kører med ca 5.  interval!}

\sub{Noter}

\p{Terminal på cluster: brug Jupyter notebooks terminalen. Herefter har du en fin terminal på cluster noden}

\p{SSH til cluster: eller gør det på den klassiske metode via SSH til clusterens 'masternode' via}

	\displaycode{> ssh -p 443 gpucluster.st.lab.au.dk}

\p{og herfra videre til 'noder' via ssh node 1 til node 5, f.eks.}

	\displaycode{> ssh node3}

\p{Brug ikke 'masternode'en til udregninger, kun node 1 til 5!}

\p{Sæt gerne jeres SSH certifickater på, så i slipper for login/password.}

\p{Password kan kun ændres via SSH og \code{> passwd}}

\p{Se Hvad der kører på CPU}

	\displaycode{! top -n1}

\p{Se Hvad der kører på GPU}

	\displaycode{! nvidia-smi}

\p{Kill din egne processer}

	\displaycode{! kill -9 <pid>}

\p{eller}

	\displaycode{! pkill <procesnavn>}

%%%%%%%%%%%%%%%%%%%%%%%%%%%%%%%%%%%%%%%%%%%%%%%%%%%%%%%%%%%%%%%%%%%%%%%%%%%%%%%
%%%%%%%%%%%%%%%%%%%%%%%%%%%%%%%%%%%%%%%%%%%%%%%%%%%%%%%%%%%%%%%%%%%%%%%%%%%%%%%

CONTENT L00

%%%%%%%%%%%%%%%%%%%%%%%%%%%%%%%%%%%%%%%%%%%%%%%%%%%%%%%%%%%%%%%%%%%%%%%%%%%%%%%
%%%%%%%%%%%%%%%%%%%%%%%%%%%%%%%%%%%%%%%%%%%%%%%%%%%%%%%%%%%%%%%%%%%%%%%%%%%%%%%

%\header{Forberedelse inden kursusstart}

\sub{Formål}

\p{\i{Gruppe tilmelding:} tilmeld dig til en ITMAL gruppe (find link i Brightspace!).}

\p{\i{Installation}: de obligatoriske værktøjer til ITMAL inden kursusstart (dvs.
L01).}

\p{\i{Forberedelse til L01:} Hent GIT repositories til litteraturen [GITHOML], prøv at
kører et par Jupyter Notebooks [JPYNB], og læs mere om pythons NumPy
bibliotek.}

\p{\i{Ekstra materiale til forberedelse:} optionelle python opgaver, hvis du vil sætte
dig mere ind i sproget.}

\sub{Installation}

\ul{
	\li{Installer Anaconda på din PC:}
	\ul{
	\li{\linkex{www.anaconda.com/products/individual, https://www.anaconda.com/products/individual}}
		\li{vælg 'Download' (downloader direkte for Windows),}
		\li{eller vælg Linux eller Mac, 32 eller 64 bit (dit valg),} 
		\li{nværende nyeste Anaconda3 version er \b{2021.05}}
	}
	\ul{
		\li{ALTERNATIV 1:}
		\ul{
			\li{brug vores ASE GPU Cluster som jupyter hub server,}
			\li{se info in [KURSUSINFOGPU].}
		}
		\li{ALTERNATIV 2:}
		\ul{
			\li{Lav en konto på Google's Colaboratory,}
			\li{\link{https://colab.research.google.com}}
		}
	}
	\li{Test at du kan køre jupyter notebooks [JYPYNB] fra [GITHOML], prøv f.eks. \ipynb{index.ipynb}}
}

\sub{Forberedelse til Lektion 01}

\ul{
	\li{Læs materiale i [KURSUSINFORMATION],}
	\li{få fat i litteratur til kurset,}
	\li{clone [GITHOML] til din egen PC, se how-to under [KURSUSFORKORTELSER].}
	\li{skim denne tutorial igennem:}
	\displaystyle{\em{§ Scientific Python tutorials:} NumPy, \ipynb{tools_numpy.ipynb}, [GITHOML]
		
		[BR][BR]
		
		Læs blot, hvad du finder relevant så som 'iteration', men spring blot over
		emner, der er for komplekse eller for 'pythoniske', så som 'Stacking arrays' og
		'QR decomposition'. 
	}
}

\sub{Note vdr. kildekritik og 'informations-overload'}


\p{Vi vil i dette kurset tit kunne blive overvældet af for meget ekstern
information (informations-overload), så du skal danne dig en metode til at
kunne selektere og navigere i materialet.}

\p{Vi vil primært holde os til [HOML], [GITHOML] og Scikit-learn, med en note
om, at nettet flyder over med ekstra (til tider ubrugelig/ufiltreret)
information: en kildekritiks holdning er vigtig!} 

\sub{Ekstra materiale til forberedelse}

\p{Hvis du har brug for at opfriske dit lineær algebra matematik eller er helt ny
til python, så kan du f.eks.  læse/skimme følgende notebooks, i prioriteret
rækkefølge:}

\ol{
	\li{[OPTIONAL] python og vectors/matrices math:            [BR] \indent{\ipynb{math_linear_algebra.ipynb}        [GITHOML],}}
	\li{[OPTIONAL] python og grafisk plotting:                 [BR] \indent{\ipynb{tools_matplotlib.ipynb}           [GITHOML],}}
	\li{[OPTIONAL] ekstra, Python og dataværktøjet 'Pandas':   [BR] \indent{\ipynb{tools_pandas.ipynb}               [GITHOML],}}
	\li{[OPTIONAL] ekstra, mest for de matematik intereserede: [BR] \indent{\ipynb{math_differential_calculus.ipynb} [GITHOML].}}
}

\p{Pandas er et meget populært databehandlingsværktøj, men det
bruges/introduceres dog ikke formelt i dette kursus (du er velkommen til selv
at undersøg/bruge det).}

%%%%%%%%%%%%%%%%%%%%%%%%%%%%%%%%%%%%%%%%%%%%%%%%%%%%%%%%%%%%%%%%%%%%%%%%%%%%%%%
%%%%%%%%%%%%%%%%%%%%%%%%%%%%%%%%%%%%%%%%%%%%%%%%%%%%%%%%%%%%%%%%%%%%%%%%%%%%%%%

CONTENT L01

%%%%%%%%%%%%%%%%%%%%%%%%%%%%%%%%%%%%%%%%%%%%%%%%%%%%%%%%%%%%%%%%%%%%%%%%%%%%%%%
%%%%%%%%%%%%%%%%%%%%%%%%%%%%%%%%%%%%%%%%%%%%%%%%%%%%%%%%%%%%%%%%%%%%%%%%%%%%%%%

%\header{Introduktion}

\sub{Formål}

Denne lektion har til formål at give indledende information om kurset.  Dvs. 
at vi præsentere de formelle rammer vdr.

\ul{
	\li{ITMAL gruppetilmelding,}
	\li{opgavesæt og journalafleveringer,}
	\li{eksamensform,}
	\li{Blackboard opbygning og fildeling.}
}

\p{Herefter vil vi præsentere machine learning [ML] som koncept overordnet, og
kort ridse lektionsplanen for kurset op.}

\p{Software til brug for kurset introduceres og skal installeres på jeres PC'er,
se 'L00: Forberedelse' for en installationsguide.  Vi anvender python
distributionen anaconda og i henter og installere den sidste nye version.  På
klassen vil der blive givet en kort demo af jupyter notebooks, dvs.  et at de
udviklingsværktøjer til python vi vil bruge.}

\p{Vi kigge på Scikit-learn, det primære eksterne web-sted vi vil bruge i kurset,
samt forsøge os med et par små programmer i python.}

\p{Til slut kigger vi på supervised learning og at kunne predicte
'life-satisfactory' via demo projektet i [HOML], og vi ser på pythons modul- og
klassebegreber (modules, classes), så vi kan genbruge kode i senere
lektioner..}

\sub{Indhold}

\ul{
	\li{Formelle rammer vdr. kurset.}
	\li{Eksamensform, godkendelsesfag via:}
	\ul{
		\li{et sæt obligatoriske skriftlige gruppe-journaler med afleveringsdeadlines,}
		\li{en poster-session, med aflevering af poster og mundtlig præsentation af poster,}
		\li{en mundtlig gennemgang af den sidste journal med alle medlemmer i ITMAL gruppen, samt evaluering af hver gruppemedlems
		bidrag.}
		\displaystyle{\b{\style{color: #ff3333, => Endelig godkendelse af kurset sker på en samlet vurdering af de tre punkter ovenfor.}}}
	}
	\li{Læringsmål.}
	\li{Litteratur.}
	\li{Intro til software, der bruges i ITMAL:}
	\ul{
		\li{python generelt (link til mini python intro: \link{[HOME]/L01/demo.ipynb},}
		\li{anaconda python distribution:}
		\ul{
			\li{jupyter notebooks,}
			\li{spyder developer environment.}
		}
		\li{Scikit-learn,}
		\li{opgave med python modul og klasser.}
	}
	\li{Intro til machine learning:}
	\ul{
		\li{Supervised learning (regression): 'life-satisfactory' [HOML].}
	}
}

\sub{Litteratur}

	\displaystyle{§ Preface, p. xv [HOML] (eksklusiv fra Using Code Examples...og resten af intro	kapitlet)}
	
	\displaystyle{§ 1 The machine Learning Landscape [HOML]}

	\displaystyle{§ 2 End-to-End Machine Learning Project [HOML]}

\p{Dette kapitel indeholder mange nye koncepter og en del kode.  Vi vender
senere tilbage til kapitlet senere, så læs det og prøv at danne dig et overblik
(dvs. nærlæs ikke).}

\p{Når du har installeret anaconda (se L00):}

	\displaystyle{§ Scientific Python tutorials: NumPy}
	
	\displaystyle{tools_numpy.ipynb [GITHOML]}

\p{Læs blot, hvad du finder relevant så som 'iteration', men spring blot over
emner, der er for komplekse eller for 'pythoniske', så som 'Stacking arrays' og
'QR decomposition'.}

\sub{Opgaver}

\p{Forberedelse inden lektionen}

\ul{
	\li{Meld dig ind i en ITMAL working-group [G].}
	\li{Følg installation processen givet i lektion nul ('L00: Forberedelse').}
	\li{Læs pensum.}
}

\sub{På klassen}

\ul{
	\li{Diskussion om ML (indlejret i forelæsningen).}
	\li{Opgave: intro.ipynb}
	\li{HUSK DATA til intro'en (download og udpak så "dataset" dir ligger sammen med intro.ipynb): \link{[HOME]/L01/datasets.zip}}
	\li{Opgave: modules_and_classes.ipynb}
}

\sub{Optionelle opgaver}

\p{Se 'Ekstra materiale til forberedelse' i lektion 'nul', specielt hvis du har
brug for en python og lineær algebra kick-start.}

\sub{Slides}

\displaystyle{
	\link{[HOME]/L01/lesson01.pdf}
}

END
