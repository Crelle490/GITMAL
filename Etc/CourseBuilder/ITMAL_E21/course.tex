COURSE
DEFS
	[HOMEHTML] https://itundervisning.ase.au.dk/ITMAL_E21/Html
	[HOME]     https://itundervisning.ase.au.dk/ITMAL_E21
	[FIGS]     https://itundervisning.ase.au.dk/ITMAL_E21/Html/Figs
	[HOML]     <span style='font-family: courier new, courier;'>[HOML]</span>
	[GITMAL]   <span style='font-family: courier new, courier;'>[GITMAL]</span>
	[GITHOML]  <span style='font-family: courier new, courier;'>[GITHOML]</span>
	[JPYNB]    <span style='font-family: courier new, courier;'>[JPYNB]</span>
	[OPTIONAL] (OPTIONEL)

	[KURSUSINFORMATION]  <a href='https://brightspace.au.dk/d2l/le/lessons/27524/units/244588'>kursusinformation</a>
	[KURSUSFORKORTELSER] <a href='https://brightspace.au.dk/d2l/le/lessons/27524/topics/254943' rel='noopener' target='_blank'>kursusinformation | kursusforkortelser</a>
	[KURSUSINFOGPU]      <a href='https://brightspace.au.dk/d2l/le/lessons/27524/topics/244596' rel='noopener' target='_blank'>kursusinformation | GPU Cluster</a>

	[SLIDES(arg)]       slide arg

	[BR] <br>

%%%%%%%%%%%%%%%%%%%%%%%%%%%%%%%%%%%%%%%%%%%%%%%%%%%%%%%%%%%%%%%%%%%%%%%%%%%%%%%
%%%%%%%%%%%%%%%%%%%%%%%%%%%%%%%%%%%%%%%%%%%%%%%%%%%%%%%%%%%%%%%%%%%%%%%%%%%%%%%

CONTENT Litteratur

\sub{Hands-on Machine Learning [HOML]}

\dl{
	\dd{\img{[FIGS]/book_homl.jpg, Hands-on Machine Learning with Scikit-Learn (front image)}}
	\dd{\i{Hands-on Machine Learning with Scikit-Learn, Keras, and TensorFlow:[BR] Concepts, Tools, and Techniques to Build Intelligent Systems}}
	\dd{[BR]}
	\dd{Aurélien Géron}
	\dd{O'Reilly / Wiley, 2019, 2.ed.}
	\dd{ISBN: 9781492032649}
	\dd{\link{O'Reilly link,https://www.oreilly.com/library/view/hands-on-machine-learning/9781492032632/}}
	\dd{[BR]}
	\dl{
		\dt{\i{NOTE 1:}}
			\dd{Dette er anden udgave (Second Edition/2.ed) af Géron's \i{Hands-on},
			undgå at bruge førsteudgaven, idet den benytter TensorFlow direkte i stedet for
			Keras, og desuden har flere mangler.}

		\dt{\i{NOTE 2:}}
			\dd{I PDF udgaven (Early Release, June 2019, 2019-04-22: Fifth Release)
			svarer sidetal og nogle kapitler ikke til den officielle bog udgave ovenfor!}
	}
}

\sub{Deep Learning [DL]}

\dl{
	\dd{\img{[FIGS]/book_dl.jpg, Deep Learning (front image)}}
	\dd{\i{Deep Learning}}
	\dd{[BR]}
	\dd{Ian Goodfellow, Yoshua Bengio, Aaron Courville}
	\dd{The MIT Press}
	\dd{November 18, 2016}
	\dd{Hardcover: 775 pages}
	\dd{ISBN-10: 0262035618}
	\dd{ISBN-13: 978-0262035613}
	\dd{\link{http://www.deeplearningbook.org/}}
	\dd{[BR]}
	\dl{
		\dt{\i{NOTE:}}
			\dd{Ikke obligatorisk, kun få afsnit og figurer bruges herfra.
			(Bog god til videregående Neural Netværks-teori
			og meget brugt i ML sammenhænge.)}
	}
}


%%%%%%%%%%%%%%%%%%%%%%%%%%%%%%%%%%%%%%%%%%%%%%%%%%%%%%%%%%%%%%%%%%%%%%%%%%%%%%%
%%%%%%%%%%%%%%%%%%%%%%%%%%%%%%%%%%%%%%%%%%%%%%%%%%%%%%%%%%%%%%%%%%%%%%%%%%%%%%%

CONTENT Kursusforkortelser

\dl{
	\dt{[\b{AI}]:}
	\dd{Artificial intelligence (kunstig intelligens). Generelt bruges udtrykket ML i stedet for AI i kurset.}

	\dt{[\b{CNN}]}
	\dd{Convolutional Neural Network(s), undersort at NN's, primært til billebehandling.}

	\dt{[\b{DL}]}
	\dd{Enten bare Deep Learning eller Deep Learning bogen af Ian Goodfellow, et. al.}

	\dt{[\b{G}]}
	\dd{Group, ITMAL øvelsesgruppe.}

	\dt{[\b{GITHOML}]}
	\dd{
		\dl{
			\dt{Github repository til [HOML],}
			\dd{\link{https://github.com/ageron/handson-ml2/}}
			\dt{Clone via HTTPS}
			\dd{\code{git clone https://github.com/ageron/handson-ml2.git}}
			\dt{eller via SSH}
			\dd{\code{git clone git@github.com:ageron/handson-ml2.git}}
		}
	}

	\dt{[\b{GITMAL}]}
	\dd{
		\dl{
			\dt{Git repository for ITMAL,}
			\dd{\link{https://gitlab.au.dk/au204573/GITMAL/}}
			\dt{Clone via HTTPS}
			\dd{\code{git clone https://gitlab.au.dk/au204573/GITMAL.git}}
			\dt{eller via SSH}
			\dd{\code{git clone git@gitlab.au.dk:au204573/GITMAL.git}}
		}
	}

	\dt{[\b{HOML}]}
	\dd{Hands-on Machine Learning af Aurélien Géron, hovedlitteratur til dette kursus. For klarhedens skyld udtales 'HOML' som Holm i Brian Holm.}
	\dd{
		[BR]\img{[FIGS]/brian_holm.jpg, Brian Holm (foto fra cdn-ctstaging.pressidium.com)}
		[BR]\cite{https://cdn-ctstaging.pressidium.com/wp-content/uploads/2020/12/CORVOS_00000365-066.jpg}
	}

	\dt{[\b{ITMAL}]}
	\dd{IT Machine Learning, kursusnavnet.}

	% [J1, J2, .. JN]: En journal opgave/aflevering, f.eks. "Journal 1" (J1). 'Journaler' erstattes af 'opgave afleveringer', O1, O2, osv.

	\dt{[\b{JPYNB}]}
	\dd{Jypyter Python NoteBook, dvs. Notebook applikationen eller en notebook kildetekst fil (med endelsen .ipynb).}

	\dt{[\b{ML}]}
	\dd{Machine Learning, det generelle koncept.}

	\dt{[\b{NN}]}
	\dd{Neural Network(s). Normalt forstået som fully-connected neurale netværk (se også CNN).}

	\dt{[\b{O1, O2, O3, O4}]}
	\dd{En opgaveaflevering, f.eks. O1 for opgaveaflevering 1.} %(opgave afleveringer hed tidligere journaler).

	%\dt{[RNN:]}
	%\dd{Recurrent Neural Network(s), undersort at NN's, men med indbygget \i{hukommelse}. }


	\dt{[\b{SG}]}
	\dd{Super-gruppe, bestående af tre eller fire Grupper [G]'s.}

	\dt{[\b{Q}]:}
	\dd{Et specifikt spørgsmål (Questions) i en journal opgave, ala Qc for opgave 'c' i et journal spørgsmål.}
}


%%%%%%%%%%%%%%%%%%%%%%%%%%%%%%%%%%%%%%%%%%%%%%%%%%%%%%%%%%%%%%%%%%%%%%%%%%%%%%%
%%%%%%%%%%%%%%%%%%%%%%%%%%%%%%%%%%%%%%%%%%%%%%%%%%%%%%%%%%%%%%%%%%%%%%%%%%%%%%%

CONTENT Dokumentation og links

\sub{Web sites}

Primære

\dl{
    \dt{[GITHOML]:}
	    \dd{\link{https://github.com/ageron/handson-ml2/}}
    \dt{Scikit-learn:}
    	\dd{\link{https://scikit-learn.org/stable/}}
    \dt{Keras:}
    	\dd{\link{https://keras.io/}}
}

Sekundære

\itemize{
    \item{\link{Jupyter: jupyter-notebook.readthedocs.io/en/stable,  https://jupyter-notebook.readthedocs.io/en/stable/}}
    \item{\link{Anaconda Cloud: anaconda.org,                        https://anaconda.org/}}
   	\item{\link{Tensorflow: www.tensorflow.org,                      https://www.tensorflow.org/}}
}

Datakilder

\itemize{
    \item{\link{Kaggle datasæts: www.kaggle.com, https://www.kaggle.com/}}
    \item{\i{Sign in with your email} => genbrug gerne min konto, og undgå tidssplid:
    	\itemize{
           \item{user: cef@ase.au.dk}
           \item{password: test123}
        }
	}
}

Dokumentation

\itemize{
    \item{Brug den indbyggede hjælp i [JPYNP]}
    \img{[FIGS]/Screenshot_jupyter_help.png,}
}

Guides etc.

\itemize{
    \item{\link{A Quick Python intro (short),      https://www.w3schools.com/python/python_intro.asp}}
    \item{\link{A Python tutorial (not so short!), https://docs.python.org/3/tutorial/}}
    \item{Jupyter shortcuts quick guide XXX}
    \item{Scikit-learn reference XXX}
    \item{Scikit-learn cheat sheet XXX}
}


%%%%%%%%%%%%%%%%%%%%%%%%%%%%%%%%%%%%%%%%%%%%%%%%%%%%%%%%%%%%%%%%%%%%%%%%%%%%%%%
%%%%%%%%%%%%%%%%%%%%%%%%%%%%%%%%%%%%%%%%%%%%%%%%%%%%%%%%%%%%%%%%%%%%%%%%%%%%%%%

CONTENT GPU Cluster

Der er adgang til en GPU baseret server i forbindelse med kurset.  Serveren består af en
'master' som kan tilgås via

\dl{
	\dd{\link{http://gpucluster.st.lab.au.dk/}}
}

og fem 'slave' noder med GPU'er via

\dl{
	\dd{\link{http://gpucluster.st.lab.au.dk/jhub1}}
	\dd{\link{http://gpucluster.st.lab.au.dk/jhub2}}
	\dd{\link{http://gpucluster.st.lab.au.dk/jhub3}}
	\dd{\link{http://gpucluster.st.lab.au.dk/jhub4}}
	\dd{\link{http://gpucluster.st.lab.au.dk/jhub4}}
	\dd{\link{http://gpucluster.st.lab.au.dk/jhub5} (GPU 3090 og ny 4GHz CPU, 3090 har problemer med Tensorflow)}
}

som frit kan benyttes.

\p{Adgang kræver at i er på EDUROAM eller VPN/au access.}

\p{Der er pt. ingen load-balancing på de fire noder så se på gpucluster hjemmesiden, hvilke noder der er mindst belastede (via TOP
dataene).}

\sub{Brug}

\p{Alle grupper har fået deres egen konto, og f.eks. så logger Grp 09 ind som:}

\dl{
	\dd{Login: f21mal09}
	\dd{Password: f21mal09_123}
}

\p{Dvs. brugernavn/login f21malXX hvor XX er jeres ITMAL gruppe og password sammen som brugernavn med _123
tilføjet.}

\sub{Quick Guide}

\dl{
	\dt{GIT via Jupyter:}
		\dd{I kan clone git repositoret via en '!'-shell commando i Jupyter notepad'en}
		\dd{\code{! git clone https://cfrigaard@bitbucket.org/cfrigaard/itmal}}
		\dd{og så herefter pull'e via}
		\dd{\code{! cd itmal && git pull}}

	\dt{PYTHONPATH}
		\dd{NOTE: PYTHONPATH er IKKE sat (som vi gjorde i L03/modules_and _classes.ipynb), men kan simuleres via}
		\dd{\code{import sys,os sys.path.append(os.path.expanduser('~/itmal'))}}
}

\p{Hvis du kloner [GITMAL] til itmal som ovenfor er der nu automatisk sat en path op til libitmal, prøv det!}

\sub{GPU Hukommelse}

\p{Ved brug af Keras+GPU allokeres automatisk al GPU hukommelse.  Når vi er flere brugere skal i derfor indsætte
følgende i starten af jeres Keras/Tensorflow Jupyter notebook kode:}

	\displaycode{
		import tensorflow as tf
		from keras.backend.tensorflow_backend import set_session
		config = tf.ConfigProto()
		config.gpu_options.per_process_gpu_memory_fraction = 0.05
		config.gpu_options.allow_growth=True
		set_session(tf.Session(config=config))
	}

eller blot

	\displaycode{
		from libitmal import kernelfuns as itmalkernelfuns
		itmalkernelfuns.EnableGPU()
	}

\p{så allokeres kun en brøkdel af GPU hukommelsen!  Det ser ud til at growth=True ikke virker, så sæt
per_process_gpu_memory_fraction op hvis i har behov.}

\p{Der kører nu et automatisk startup-script når i logger ind/åbner en notebook. Se}

	\displaycode{/home/shared/(??)/startup/00_init.py}

\p{der kører StartupSequence_SetPath() og StartupSequence_EnableGPU(), den sidste funktion med følgende default
parametre}

	\displaycode{def StartupSequence_EnableGPU(gpu_mem_fraction=0.05, gpus=None, cpus=None)}

\p{Bemærk at jeres jupyter server, beholder all hukommelse, også når i logger af...kun "stop my server"/"start server"
frigiver!}

%\p{Jeg vil slå alle processer ned, der har allokeret over ca.  4Gb GPU hukommelse eller har kørt i en uge...det er en
%automatisk process, der kører med ca. 5.  interval!}

\sub{Noter}

\p{Terminal på cluster: brug Jupyter notebooks terminalen. Herefter har du en fin terminal på cluster noden}

\p{SSH til cluster: eller gør det på den klassiske metode via SSH til clusterens 'masternode' via}

	\displaycode{> ssh -p 443 gpucluster.st.lab.au.dk}

\p{og herfra videre til 'noder' via ssh node 1 til node 5, f.eks.}

	\displaycode{> ssh node3}

\p{Brug ikke 'masternode'en til udregninger, kun node 1 til 5!}

\p{Sæt gerne jeres SSH certifikater på, så i slipper for login/password.}

\p{Password kan kun ændres via SSH og \code{> passwd}}

\p{Se Hvad der kører på CPU}

	\displaycode{! top -n1}

\p{Se Hvad der kører på GPU}

	\displaycode{! nvidia-smi}

\p{Kill din egne processer}

	\displaycode{! kill -9 <pid>}

\p{eller}

	\displaycode{! pkill <procesnavn>}



%%%%%%%%%%%%%%%%%%%%%%%%%%%%%%%%%%%%%%%%%%%%%%%%%%%%%%%%%%%%%%%%%%%%%%%%%%%%%%%
%%%%%%%%%%%%%%%%%%%%%%%%%%%%%%%%%%%%%%%%%%%%%%%%%%%%%%%%%%%%%%%%%%%%%%%%%%%%%%%

CONTENT Kriterier for O4

\sub{Kriterier for O4 rapport}

\p{Til den sidst aflevering, O4, skal i præsentere jeres arbejde med jeres eget
projekt.}

\p{I skal samle arbejdet i rapport-form i en PDF, dog maks.  15 sider.  Dertil
kan komme evt.  bilag/appendices.  Hvis i kan lave en velformateret/letlæst
Jupyter Notebook, kan i også aflevere direkte i dette format i stedet for
PDF.}

\p{Der skal være en oversigt over gruppemedlemmernes bidrag til rapporten.
Dvs.  at i skal lave en tabel, der viser, hvilke afsnit (eller sider) de
enkelte medlemmer primært har stået for.  Er rapportarbejdet evt.  fordelt
ligeligt mellem gruppemedlemmerne, skriv i at alle har deltaget i alle dele.}

\p{Herudover følger O4 rapporten de almindelige formelle journal krav fra
O1+2+3, på nær disse to punkter}

\itemize{
    \item{Overskrifter på de opgaver, der svares på [..] (MUST)}
    \item{Ingen cut-and-paste af tekst fra opgaveteksten.}
}

\p{som for O4 ikke giver mening (der er ingen overskrifter eller opgavetekst
at kopiere fra).}

\p{Rapporten skal opsummere jeres arbejde med jeres problemstilling i et
'end-to-end' perspektiv (hent inspiration i \i{§ 2 End-to-End Machine Learning
Project} [HOML]).}

\p{Følgende hovedpunkter skal beskrives}

\itemize{
	\item{\i{Problemstilling:} hvilket problem forsøger i at løse med ML?}

    \item{\i{Datasæt:} hvilke data arbejder i med, hvilke features, hvor
    stammer det fra osv.}

        \subitem{Tag udgangspunkt i eller genbrug jeres analyse/tekst fra
       	 afleveringen "Beskrivelse af eget slutprojekt" fra O2 (dvs. i gerne
       	 må selv-plagiere her!).}

    \item{\i{Valg af ML algoritme(r):} beskriv f.eks.}

        \subitem{hvilken grundliggende ML algoritmeklasse(r) valgte i
        (supervised/unsupervised, regression/klassifikation)?}

        \subitem{hvilke kriterier lå til grund for jeres model selection?}

        \subitem{hvorfor er jeres valgte algoritme god til netop jeres dataset?}

        \subitem{hvad er fordele og ulemper ved den (eller de) valgte
        algoritme(r), f.eks.  kompleksitet?}

	\item{\i{ML data processering:} beskrivelse af hvordan i splitter data i
        train-test set, preprocessere, træner og tester.}

        \subitem{evt. brug af Scikit-learn pipelines i processings-steps
        (direkte brug af Scikit-learn pipelines: COULD!}

	\item{\i{Performance metrics:} beskrivelse af, hvordan i måler
	'effektiviteten' af jeres træning og test.}

    \item{\i{Under- og overfitting:} hvordan sørger i for, at jeres system ikke
    under- eller overfitter på jeres data?}

    \item{\i{Optimeringer og forbedringer:} hvordan har i forbedret jeres
    system via optimeringsparametre (optimizers, regulizers) eller afsøgning af
    hyperparameterrummet?}
}

\p{I har frie hænder til at lave jeres egen disposition over opgaven, dvs.  at
de nævnte kriterier ovenfor kan komme i den rækkefølge i selv vælger.}

\p{Sørg herudover for at have en indledning og konklusion på opgaven.}

\subsub{NOTE: vdr. sidetal til O4}

\p{Der står maks.  15 sider til O4,  og det er ment som at hele O4 skal holdes
til (ca.) 15 sider inklusive figurer og tabeller.}

\p{Det er ikke ment direkte (og formelt) som ’normalsider’ med ”antal anslag,
figurer tæller ikke med..” osv.  Så hold jer til ca.  15 sider med en fornuftig
font.}

\p{De 15 siders grænser er primært indført for ikke at gøre O4 for omfattende i
projekt/afleverings tid.}


%%%%%%%%%%%%%%%%%%%%%%%%%%%%%%%%%%%%%%%%%%%%%%%%%%%%%%%%%%%%%%%%%%%%%%%%%%%%%%%
%%%%%%%%%%%%%%%%%%%%%%%%%%%%%%%%%%%%%%%%%%%%%%%%%%%%%%%%%%%%%%%%%%%%%%%%%%%%%%%

CONTENT Journal afleveringsformat

\sub{Journal afleveringsformat [O1,O2,O3]}

\p{I skal uploade opgaverne / journalerne i PDF format.}

\p{Journalen skal være en 'komplet' aflevering, dvs.  at i gerne må vedhæfte
bilag, men journal skal indeholde al nødvendig information vdr.  print/test af
kode.  Journal-resultater bør dokumenteres til den grad de kan gentages (af
andre eller jer selv senere..!).}

\p{Dvs.  at journalen skal bestå af forklarende tekst, figurer, kode og
test-output, så den danner et forståeligt billede af, hvordan i har løst
opgaven.}

\p{Men i behøver ikke at inkludere alt kode eller test-output, kun det
vigtigste, så man via journalen kan forstå og gentage jeres 'forsøg'.}

\p{Som alternativ til PDF kan man aflevere i direkte i Jupyter Notebooks format,
se note nederst, dog anbefales PDF over \code{.ipynb}.}

\sub{Almindelige formelle journal krav}

\itemize{
	\item{Forside med information om opgaven (MUST)}

        \subitem{kursus og opgaveafleveringsnavn,}
        \subitem{dato,}
        \subitem{ITMAL gruppe nummer,}
        \subitem{liste af studerende, der bidrager (med studienummer).}

	\item{Sidetal på alle sider (PDF aflv.: MUST;  Jupyter Notebooks aflv.:
	DONT or COULD).}

	    \subitem{Evt. indholdsfortegnelse (COULD).}

	\item{Overskrifter på de opgaver, der svares på, ala "Qa - Lineær
	regressions parametre og R2 scoren." (MUST)}

    \item{Ingen cut-and-paste af tekst fra opgaveteksten (opsummer eller
    omformuler i stedet opgaven med jeres egen ord i stedet) (MUST).}

        \subitem{Ingen direkte genbrug af overskrifter ala "Qa - The Θ
        parameters and the R2 score"(MUST), omformuler overskrifterne og gør
        dem til jeres egen overskrifter, sammenfald af tekststykker dog OK.}

        \subitem{Genbrug og cut-and-paste af formler og figurer fra
        opgaveteksten dog OK!  (COULD)}

        \subitem{Essentielle plots og formler bør kopieres fra opgaveteksten og
        indsættes i jeres journal (SHOULD).}

        \subitem{Genbrug af enkelte kode stumper fra opgaveteksten også OK
        (COULD).}

    \item{Indsættelse af relevant Pyton-kode og output i figurer eller lister,
    samt ref.  til disse i teksten,}

        \subitem{Sørg for at kode er læsbar, specielt ved rå screendumps
        (MUST).}

        \subitem{Forklarende figurtekster til alle figurer/lister (MUST).}

    \item{En 'fornuftig' forklarende journaltekst, så resultater kan genskabes
    (MUST).}

    \item{Vælg mellem dansk eller engelsk journal sprog (eller norsk eller
    	 svensk hvis du/I er nordisk(e) studerende) (MUST).}

    	 \subitem{Skriv kun på eet sprog igennem hele journalen, dvs.  bland
    	 ikke f.eks.  dansk og engelsk (MUST).}

    	 \subitem{Brug af MANGE engelske fagudtryk, begreber og låneord i den
    	 danske tekst: no-problem!}
}

\sub{Journal skrivning}

\p{Husk at det er en journal, ikke en rapport, der afleveres.  Tekst og figurer
skal være læsbare, og indeholde tilstrækkelig information, så resultater kan
genskabes, men det behøver ikke at blive en lang udførlig rapport.}

\p{Foruden de alm.  formelle journal-krav ovenfor, skal der svares på alle opgaver
undervejs.  Skriv i journal, hvis i ikke kan komme igennem en delopgave, og
beskriv omhyggeligt, hvad i har forsøgt og undersøgt.  Ved evt.
gen-aflevering}

\p{Mangler der en væsentlig mængde svar til underopgaver (to eller flere) eller er
journal for mangelfuldt udført (f.eks.  ved generelt manglende tekst eller at
en eller flere punkter i de formelle krav er udeladt), sendes journal til
gen-aflevering.}

\sub{Genafleveringsfrist}

\p{Der er som udgangspunkt ingen tidsfrist for en gen-aflevering.  Dog skal
alle journaler være godkendt til en kursusdeadline for at kurset kan
godkendes/bestås (lige omkring O4 afleveringen).}

\subsub{Note vdr. Jupyter Notebooks aflevering}

\p{Som alternativ til en almindelig PDF aflevering kan i også
aflevere direkte i \code{.ipynb} format.  Dvs.  at i kan lave en Jupyter
Notebook som kvalitetsmæssigt er lig en PDF afleveringen.}

\p{Med 'kvalitetsmæssigt er lig PDF' menes: overskrifter til afsnit, velformateret
tekst i markdown celler, ikke alt for store mængder kode celler direkte i
'hovedrapport', kun få korte output celler (f.eks.  ingen lange listninger af
trænings iterationer), gerne plots og andre interne (embeddede i Notebook),
evt.  eksterne figurer som html billeder (læg billeder på en server, så der er
public adgang, se html embedding metode i f.eks.  L01/intro.ipynb).  Bilag kan
placeres i slutning af Notebook, og her gælder disse kvalitetskrave ikke.}

\p{Ved aflevering i Jupyter Notebooks skal AL tekst fra opgavebeskrivelse
fjernes...og i skal skrive jeres helt egen tekst.}

\p{Brug IKKE den indbyggede PDF konverter, men aflever en kørt Notebook, og
bemærk at indlejring af eksterne billeder (png/jpg) via.  html kan være
problematisk, se min metode i opgaverne.}

\p{Bemærk at det anbefales at lave en normal PDF aflevering, da det kan være
svært af få nok typografisk 'kvalitet' i en Jupyter Notebook aflevering.}



%%%%%%%%%%%%%%%%%%%%%%%%%%%%%%%%%%%%%%%%%%%%%%%%%%%%%%%%%%%%%%%%%%%%%%%%%%%%%%%
%%%%%%%%%%%%%%%%%%%%%%%%%%%%%%%%%%%%%%%%%%%%%%%%%%%%%%%%%%%%%%%%%%%%%%%%%%%%%%%

CONTENT Ekstra video materiale

Offline lektioner

Nedenfor er tilføjet ekstra lektions-materiale. Dette materiale vil ikke blive gennemgået til online lektioner.

Offline lektioner:

    MMLS recap og matrix vektor notation:
        opsummering af matrix og vektor notationen fra 1. semester kurset 'matematisk modellering af lineære systemer'.
    Hypothesis funktionen og samling af data:
        Hvordan ser en ML 'black-box' ud, hvad er input, output og navngivning generelt.
    The Design matrix:
        hvordan data nemmere beskrives og behandles i matrix notation.
    Lineær regressor:
        Mere om en simple hypothesis algoritme; en lineær regressor funktion.
    Træning:
        Hvordan sættes data, hypothesis funktionen, dens interne parametre og tabs-funktionen (loss/cost/objective) sammen i en simple trænings/optimerings algoritme.


%%%%%%%%%%%%%%%%%%%%%%%%%%%%%%%%%%%%%%%%%%%%%%%%%%%%%%%%%%%%%%%%%%%%%%%%%%%%%%%
%%%%%%%%%%%%%%%%%%%%%%%%%%%%%%%%%%%%%%%%%%%%%%%%%%%%%%%%%%%%%%%%%%%%%%%%%%%%%%%


\sub{Formål} %\header{Forberedelse inden kursusstart}

\p{\i{Gruppe tilmelding:} tilmeld dig til en ITMAL gruppe (find link i Brightspace!).}

\itemize{
	\item{Antal studerende per grupper er = 3.}
	\item{Grupper med 2 eller 1}
	\item{studerende vil blive sammenlagt.  Skriv til undervisere, hvis du har en gyldig grund til at være SOLO i en gruppe.}
}

\p{\i{Installation}: de obligatoriske værktøjer til ITMAL inden kursusstart (dvs.
L01).}

\p{\i{Forberedelse til L01:} Hent GIT repositories til litteraturen [GITHOML],
prøv at kører et par Jupyter Notebooks [JPYNB], og læs mere om pythons NumPy
bibliotek.}

\p{\i{Ekstra materiale til forberedelse:} optionelle python opgaver, hvis du
vil sætte dig mere ind i sproget.}

\sub{Installation}

\itemize{
	\item{Installer Anaconda på din PC:}
		\subitem{\link**{www.anaconda.com/products/individual, https://www.anaconda.com/products/individual}}
		\subitem{vælg 'Download' (downloader direkte for Windows),}
		\subitem{eller vælg Linux eller Mac, 32 eller 64 bit (dit valg),}
		\subitem{nuværende nyeste Anaconda3 version er \b{2021.05}}
	\item{ALTERNATIV 1:}
		\subitem{brug vores ASE GPU Cluster som jupyter hub server,}
		\subitem{se info in [KURSUSINFOGPU].}
	\item{ALTERNATIV 2:}
		\subitem{Lav en konto på Google's Colaboratory,}
		\subitem{\link{https://colab.research.google.com}}
	\item{Test at du kan køre jupyter notebooks [JYPYNB] fra [GITHOML], prøv f.eks. \ipynb{index.ipynb}}
}

\sub{Forberedelse til Lektion 01}

\itemize{
	\item{Læs materiale i [KURSUSINFORMATION],}
	\item{få fat i litteratur til kurset,}
	\item{clone [GITHOML] til din egen PC, se how-to under [KURSUSFORKORTELSER].}
	\item{skim denne tutorial igennem:}
		\subitem{\em{§ Scientific Python tutorials:} NumPy, \ipynb{tools_numpy.ipynb}, [GITHOML]
		\subitem{Læs blot, hvad du finder relevant så som 'iteration', men
		spring blot over emner, der er for komplekse eller for 'pythoniske', så
		som 'Stacking arrays' og 'QR decomposition'.}
	}
}

\sub{Note vdr. kildekritik og 'informations-overload'}

\p{Vi vil i dette kurset tit kunne blive overvældet af for meget ekstern
information (informations-overload), så du skal danne dig en metode til at
kunne selektere og navigere i materialet.}

\p{Vi holder os primært til [HOML], [GITHOML] og Scikit-learn, med en note om,
at nettet flyder over med ekstra (til tider ubrugelig/ufiltreret) information:
en kildekritiks holdning er vigtig!}

\sub{Ekstra materiale til forberedelse}

\p{Hvis du har brug for at opfriske dit lineær algebra matematik eller er helt
ny til python, så kan du f.eks.  læse/skimme følgende notebooks, i prioriteret
rækkefølge:}

\enumerate{
	\item{[OPTIONAL] python og vectors/matrices math:            [BR] \indent{\ipynb{math_linear_algebra.ipynb}        [GITHOML],}}
	\item{[OPTIONAL] python og grafisk plotting:                 [BR] \indent{\ipynb{tools_matplotlib.ipynb}           [GITHOML],}}
	\item{[OPTIONAL] ekstra, Python og dataværktøjet 'Pandas':   [BR] \indent{\ipynb{tools_pandas.ipynb}               [GITHOML],}}
	\item{[OPTIONAL] ekstra, mest for de matematik interesserede: [BR] \indent{\ipynb{math_differential_calculus.ipynb} [GITHOML].}}
}

\p{Pandas er et meget populært databehandlingsværktøj, men det
bruges/introduceres dog ikke formelt i dette kursus (du er velkommen til selv
at undersøg/bruge det).}


%%%%%%%%%%%%%%%%%%%%%%%%%%%%%%%%%%%%%%%%%%%%%%%%%%%%%%%%%%%%%%%%%%%%%%%%%%%%%%%
%%%%%%%%%%%%%%%%%%%%%%%%%%%%%%%%%%%%%%%%%%%%%%%%%%%%%%%%%%%%%%%%%%%%%%%%%%%%%%%

CONTENT L01

\sub{Formål} %\header{Introduktion}

Denne lektion har til formål at give indledende information om kurset.  Dvs.
at vi præsentere de formelle rammer vdr.

\itemize{
	\item{ITMAL gruppetilmelding,}
	\item{opgavesæt og journalafleveringer,}
	\item{eksamensform,}
	\item{Blackboard opbygning og fildeling.}
}

\p{Herefter vil vi præsentere machine learning [ML] som koncept overordnet, og
kort ridse lektionsplanen for kurset op.}

\p{Software til brug for kurset introduceres og skal installeres på jeres
PC'er, se 'L00: Forberedelse' for en installationsguide.  Vi anvender python
distributionen anaconda og i henter og installere den sidste nye version.  På
klassen vil der blive givet en kort demo af jupyter notebooks, dvs.  et at de
udviklingsværktøjer til python vi vil bruge.}

\p{Vi kigge på Scikit-learn, det primære eksterne web-sted vi vil bruge i
kurset, samt forsøge os med et par små programmer i python.}

\p{Til slut kigger vi på supervised learning og at kunne predicte
'life-satisfactory' via demo projektet i [HOML], og vi ser på pythons modul- og
klassebegreber (modules, classes), så vi kan genbruge kode i senere
lektioner..}

\sub{Indhold}

\itemize{
	\item{Formelle rammer vdr. kurset.}
	\item{Eksamensform, godkendelsesfag via:}
	\itemize{
		\item{et sæt obligatoriske skriftlige gruppe-journaler med afleveringsdeadlines,}
		\item{en poster-session, med aflevering af poster og mundtlig præsentation af poster,}
		\item{en mundtlig gennemgang af den sidste journal med alle medlemmer i ITMAL gruppen, samt evaluering af hver gruppemedlems bidrag.}
			\subitem{[BR]\b{\style{color: #ff3333, => Endelig godkendelse af kurset sker på en samlet vurdering af de tre punkter ovenfor.}}}
	}
	\item{Læringsmål.}
	\item{Litteratur.}
	\item{Intro til software, der bruges i ITMAL:}
	\itemize{
		\item{python generelt (link til mini python intro: \link**{[HOME]/L01/demo.ipynb},}
		\item{anaconda python distribution:}
		\itemize{
			\item{jupyter notebooks,}
			\item{spyder developer environment.}
		}
		\item{Scikit-learn,}
		\item{opgave med python modul og klasser.}
	}
	\item{Intro til machine learning:}
	\itemize{
		\item{Supervised learning (regression): 'life-satisfactory' [HOML].}
	}
}

\sub{Litteratur}

\itemize*{
	\item{\i{§ Preface}, p. xv [HOML] (eksklusiv fra \i{Using Code Examples}...og resten af intro kapitlet)}
	\item{\i{§ 1 The machine Learning Landscape} [HOML]}
	\item{\i{§ 2 End-to-End Machine Learning Project} [HOML]}
}

\p{Dette kapitel indeholder mange nye koncepter og en del kode.  Vi vender
senere tilbage til kapitlet senere, så læs det og prøv at danne dig et overblik
(dvs. nærlæs ikke).}

\p{Når du har installeret anaconda (se L00):}

\itemize*{
	\item{\i{§ Scientific Python tutorials: NumPy}}
		\subitem*{\ipynb{tools_numpy.ipynb} [GITHOML]}
}

\p{Læs blot, hvad du finder relevant så som 'iteration', men spring blot over
emner, der er for komplekse eller for 'pythoniske', så som 'Stacking arrays' og
'QR decomposition'.}

\sub{Forberedelse inden lektionen}

\itemize{
	\item{Meld dig ind i en ITMAL working-group [G].}
	\item{Følg installation processen givet i lektion nul ('L00: Forberedelse').}
	\item{Læs pensum.}
}

\sub{På klassen}

\enumerate{
	\item{Diskussion om ML (indlejret i forelæsningen).}
	\item{\b{Opgave} (introduktion): \link**{[HOME]/L01/intro.ipynb} }
		\subitem*{HUSK DATA til intro'en (download og udpak så dataset dir ligger sammen med intro.ipynb): \link**{[HOME]/L01/datasets.zip}}
	\item{\b{Opgave} (python introduktion): \link**{[HOME]/L01/modules_and_classes.ipynb}}
}

\sub{Optionelle opgaver}

\p{Se 'Ekstra materiale til forberedelse' i lektion 'nul', specielt hvis du har
brug for en python og lineær algebra kick-start.}

\sub{Slides}

\itemize*{
	\item{\link**{[HOME]/L01/lesson01.pdf} (vil blive opdateret inden lektion)}
}


%%%%%%%%%%%%%%%%%%%%%%%%%%%%%%%%%%%%%%%%%%%%%%%%%%%%%%%%%%%%%%%%%%%%%%%%%%%%%%%
%%%%%%%%%%%%%%%%%%%%%%%%%%%%%%%%%%%%%%%%%%%%%%%%%%%%%%%%%%%%%%%%%%%%%%%%%%%%%%%

CONTENT L02

\sub{Formål} %\header{Supervised classification, Cost function, Performance metrics}

\p{Vi lægger ud med at se på datamat-læring eller ML-læring via cost
funktionen, J, dvs.  hvordan en supervised ML algoritme i princippet kunne
tænkes at fungere.  Til cost funktionen tager vi hul på et par grundliggende
lineær algebra koncepter: "norm" eller afstandsmål mellem to N-dimensionale
vektorer, og vi ser hvordan Mean-Square-Error (MSE) og Mean-Absolute-Error
(MAE) kunne indgå i J.}

\p{Herefter studerer vi begrebet supervised klassifikation dybere og bruger
Scikit-learns fit-predict interface til konkret at lave og køre ML-kode.  Vi
går i detaljen med den fundamentale train/test-split og fit/predict proces vdr.
supervised learning, og vi bruger diverse kendte små dataset, som vi også vil
benytte i resten af kurset: MNIST, Iris og Moon.}

\p{Til sidst evaluere vi en kørt ML læring ved brug af diverse performance
metrics, dvs.  metoder til generelt at kvantificere hvor 'god' (kvalitet) en
læring den pågældende algoritme har opnået på de pågældende data.}

\sub{Indhold}

\itemize{
	\item{Lineær algebra og cost funktionen, J}
	\itemize{
		\item{matricer, vektors, norms og NumPy,}
		\item{MSE, MAE,
		\item{ML læring via J}
	}
	\item{Klassifikation}
        \itemize{
	        \item{’demo’ datasæt:  MNIST, iris and moon}
    	    \item{fundamental ML supervised lærings-proces,}
       		\enumerate{
           		 \item{forbered data: shuffle, stratification, normalization}
           		 \item{train/test split}
           		 \item{træn på træningsdata}
           		 \item{evaluer på test data:  performance metrics}
			}
		}
    \item{Performance metrics}
    	\itemize{
        	\item{the accuracy paradox,}
        	\item{vigtige metrics,}
			\itemize{
				\item{accuracy}
				\item{precision_score}
				\item{recall_score}
				\item{f1_score}
				\item{confusion_matrix}
			}
		}
	}
}

\sub{Litteratur}

\itemize*{
	\item{\i{§ 2 End-to-End Machine Learning Project}, \i{Select a Performance Measure}, [HOML]}
		\subitem{Genlæs KUN \i{Select a Performance Measure} (pp.39-41)}
	\item{\i{§ 3 Classification} [HOML]}
		\subitem{Skim eller spring over: \i{The ROC Curve} (pp.97-100) \i{Multilabel Classification} og \i{Multioutput Classification} (pp.106-108).}
}

\sub{Forberedelse inden lektionen}

\itemize{
	\item{Læs litteraturen.}
}

\sub{På klassen}

\enumerate{
	\item{Almindelig forelæsning}
	\item{\b{Opgave} (cost funktionen og lineær algebra):             \link**{[HOME]/L02/cost_function.ipynb}}
	\item{\b{Opgave} (supervised learning og fit-predict interfacet): \link**{[HOME]/L02/dummy_classifier.ipynb}}
	\item{\b{Opgave} (performance metrikker):                          \link**{[HOME]/L02/performance_metrics.ipynb}}
}

\sub{Slides}

\itemize*{
	\item{\link**{[HOME]/L02/lesson02.pdf}}
}


%%%%%%%%%%%%%%%%%%%%%%%%%%%%%%%%%%%%%%%%%%%%%%%%%%%%%%%%%%%%%%%%%%%%%%%%%%%%%%%
%%%%%%%%%%%%%%%%%%%%%%%%%%%%%%%%%%%%%%%%%%%%%%%%%%%%%%%%%%%%%%%%%%%%%%%%%%%%%%%

CONTENT L03

\sub{Formål} %\header{End-to-end ML}

\p{Opsamlingslektion: vi tager et genblik på \i{§ 2 End-to-End Machine Learning
Project}, og samler op på dette brede kapitel.}

\p{Vi går først i gang med at gennemgå \i{K-fold cross-validation} (eller K-fold
CV), for derefter at bruge "\link{The Map, [HOME]/Etc/ml_supervised_map.pdf}"
til at komme igennem alle grundliggende koncepter i \i{§ 2}.}


\itemize*{
	\item{
		\link{\img{[FIGS]/ml_supervised_map.png, Supervised map image.}, [HOME]/Etc/ml_supervised_map.pdf}
		[BR]
		\i{Figur: \link*{Oversigtskortet for Supervised learning (The Map).,  [HOME]/Etc/ml_supervised_map.pdf}}
	}
}

\sub{Indhold}

\itemize{
	\item{Generel genlæsning og repetition af \i{§ 2}}
	\item{K-fold Cross-validation}
}

\sub{Litteratur}

\itemize*{
	\item{\i{§ 2 End-to-End Machine Learning Project} [HOML]}
		\subitem*{genlæsning (eksklusiv \i{Create the Workspace} og \i{Download the Data})}
	\item{\i{§ Scikit's dokumentations-side vdr. K-fold CV}}
		\subitem*{\link{https://scikit-learn.org/stable/modules/generated/sklearn.model_selection.KFold.html}}
}

\sub{Forberedelse inden lektionen}

\itemize{
	\item{Læs litteraturen,}
	\item{Een eller flere af gruppe medlemmer forbereder et mundtlig resume af \i{§ 2 End-to-End}:}
		\subitem{se detaljer i \link**{[HOME]/L03/supergruppe_diskussion.ipynb}}
}

\sub{På klassen}

\enumerate{
	\item{\b{Forberedt opgave:}     \link**{[HOME]/L03/supergruppe_diskussion.ipynb}}
		\subitem*{Supergruppe [SG] resume af \i{§ 2 End-to-End}, ca. 30 til 45 min.}
    \item{Almindelig forelæsning}
		\subitem*{ekstra materiale: \link**{[HOME]/L03/Extra/k-fold_demo.ipynb}}
    \item{\b{Opgave} (pipelines):   \link**{[HOME]/L03/pipelines.ipynb}}
   		\subitem*{Data til pipelines opgaven (bør lige i L03/Data/): \link**{[HOME]/L03/Data/itmal_l01_data.pkl}}
		\subitem*{(Du har allerede denne data-fil, hvis du pull'er fra [GITMAL])}
}

\sub{Slides}

\itemize*{
	\link**{[HOME]/L03/lesson03.pdf}
}


%%%%%%%%%%%%%%%%%%%%%%%%%%%%%%%%%%%%%%%%%%%%%%%%%%%%%%%%%%%%%%%%%%%%%%%%%%%%%%%
%%%%%%%%%%%%%%%%%%%%%%%%%%%%%%%%%%%%%%%%%%%%%%%%%%%%%%%%%%%%%%%%%%%%%%%%%%%%%%%

CONTENT L04

\sub{Formål} %\header{Regression og SGD}

\p{Vi analysere \i{supervised regressions} modeller i ML i forhold til den nu
kendte supervised klassifikation metode.}


\p{Modellen \i{lineær} regression gennemanalyseres og der angives både
analytiske og numeriske løsninger til træning af en lineær model.  (Den lineære
regressionsmodel er relativ simple, men yderst vigtig, idet den men danner
grundlag for en hel del avancerede modeller, f.eks. SVN's og NN's.)}

\p{For den numeriske løsning går vi i dybden med træningsmetoden \i{Stochastic
Gradient Decent metoden} (SGD).}

\p{Endelig sluttes af med korter introduktion til logistisk regression, der danner
grundlag for neuroner i neurale net.}

\sub{Indhold}

\itemize{
	\item{Lineær regression}
	\item{Cost function i closed-form løsning vs. numerisk løsning}
	\item{Stochastic Gradient Decent (SGD)}
	\item{Learning rates}
	\item{Logistisk regression}
}

\sub{Litteratur}

\itemize*{
	\item{\i{§ 4 Training Models} (pp.111-134) [HOML]}
		\subitem*{indtil og eksklusivt \i{Regularized Linear Models}}
}

\sub{Forberedelse inden lektionen}

\itemize{
	\item{Læs litteraturen.}
}

\sub{På klassen}

\enumerate{
    \item{Almindelig forelæsning}
    \item{\b{Opgave} (SGD): \link**{[HOME]/L04/sgd.ipynb}}
}

\sub{Slides}

\itemize*{
	\item{\link**{[HOME]/L04/lesson04.pdf}}
}


%%%%%%%%%%%%%%%%%%%%%%%%%%%%%%%%%%%%%%%%%%%%%%%%%%%%%%%%%%%%%%%%%%%%%%%%%%%%%%%
%%%%%%%%%%%%%%%%%%%%%%%%%%%%%%%%%%%%%%%%%%%%%%%%%%%%%%%%%%%%%%%%%%%%%%%%%%%%%%%

CONTENT L05

\sub{Formål} %\header{Dataanalyse}


\p{En vigtig forudsætning for at kunne vælge den rette machine learning metode
til at løse et problem er data analyse, dvs.  at få en forståelse af data ved
at se på statistikker såsom middelværdi, median, varians, og plotte data og
histogrammer.}

\p{Der kan også være støjfyldte/korrupte data som skal fjernes eller data
værdier, som mangler; det skal håndteres på fornuftig vis, ellers vil man få
forkerte/dårlige modeller efterfølgende.}

\p{\i{The Curse of Dimensionallity} diskuteres og en simple unsupervised
algoritme til dimensionalitetsreduktion, PCA, undersøges.}

\p{Da alle kerne-koncepter i supervised ML nu kendes, kan det hele konkret
sammensættes i en samlet processerings-\i{pipeline}.  Programmerings-teknisk
ser vi derfor til sidst på Scikit-learns Pipelines.}

\p{Lektionen giver endelig også et oplæg til valg af data til brug for O4
afleveringen.}

\sub{Indhold}

\itemize{
	\item{Basale termer fra statistik}
		\subitem{middelværdi, median, varians, histogram}
	\item{The Curse of Dimensionallity}
		\subitem{dimensionalitet reduktion vha. PCA modellen}
	%\item{Korrelationskoefficienter}
	\item{Pipelines}
		\subitem{Data cleaning, standardization / normalisation (feature scaling)}
	\item{Indledende analyse af data med plots}
}

\sub{Litteratur}

\itemize*{
	\item{\i{§ 8 Dimensionality Reduction} (pp.213-225) [HOML], kapitel skimmes kun, men læs grundigt:}
		\subitem*{\i{Intro} (pp. 213-214)}
		\subitem*{\i{The Curse of Dimensionality} (pp. 214-215)}
		\subitem*{\i{PCA} (pp. 219-223)}
	\item{\i{§ 2 End-to-End ML} (p.38), genlæs  \i{Pipelines} tekstboksen }
		\subitem*{(herudover der en hel del genbrug af andet materiale fra \i{§ 2})}
}

\sub{Forberedelse inden lektionen}

\itemize{
	\item{Læs litteraturen.}
}

\sub{På klassen}

\enumerate{
    \item{Almindelig forelæsning}
    \item{\b{Opgave} (dataanalyse): \link**{[HOME]/L054/dataanalyse.ipynb}}
}

\sub{Slides}

\itemize*{
	\item{\link**{[HOME]/L05/lesson05.pdf}}
}


END


Dataanalyse

Formål


Litteratur

Kap. 2  s. 47 - 70 (især de emner der forelæses omkring)




L07: Generalisering

Modelkapacitet, under- og overfitting og generalisering
Formål

Der begyndes med at se på eksisterende ML-systemer, og vi diskuterer om
løsningerne via ML-systemerne er veldesignede og velgennemtestede.  Herudover
kommer der en kort "filosofisk" diskussion af ML overordnet set...(kræver ingen
formel forberedelse fra jeres side)!

Herefter gennemgår vi koncepterne under- og overfitting, der har sammenhæng med
en ML-models såkaldte kapacitet og dens evne til at generalisere.

    Generelt om ML-systemer, og klasse diskussion vdr. ML
    Model Capacity
    Under/overfitting
    Generalization Error

Der relateres til træning af NN's i \i{§ 11} i [HOML], som vi læser videre i til
næste lektion...

Litteratur

\i{§ Overfitting the Training Data} og \i{Underfitting the Training Data},
pp.27-29 [HOML] i \i{§ 1} (genlæsning).

\i{§ Polynomial Regression} og \i{Learning Curves}, pp128-134 [HOML] i \i{§
4}, \i{Training Models}.

\i{§ Training Deep Neural Networks} til og eksklusivt \i{Batch Normalization},
pp 331-338 [HOML] i \i{§ 11}.

[OPTIONAL]: læs videre i \i{§ 11}.  Næste gang går vi i dybden med \i{Fast
Optimizers}, og \i{Avoiding Overfitting Through Regularization} (mens vi
springer en del af de meget tekniske underkapitler over i \i{§ 11}).

[OPTIONAL]: læs  \i{GPT-3 er ikke stærk AI} på Version2.dk

Opgaver Forberedelse

inden lektionen

Læs litteraturen
På klassen

        Demos + diskussion vdr ML (ingen forberedelse).
        Opgave:
            capacity_under_overfitting.ipynb Click for more options
        Opgave:
            generalization_error_v2.ipynb (v2 fixer krydshenvisningsfejl)

Slides

lesson07.pdf




L06: Neurale Net

Kunstige neurale netværk
Attached Files:

    File L06_files.zip Click for more options (749.237 KB)

Formål

Kunstige neurale netværk (Artificial Neural Networks, ANN) er en af de mest
kendte machine learning modeller og har en historisk baggrund med inspiration i
hjernen.  Vi skal se på modellen for det allermest udbredte neural netværk
(Multi-Layer Perceptron) og se hvorledes det blot er en udvidelse af lineær og
logistisk regression.  Vi taler også lidt om træning af neurale netværk, som
igen er basalt set magen til de tidligere metoder (gradient descent-baseret) vi
har set.

Vi starter med øvelser i Scikit-learn framework, da det er simplere og
velkendt.  I senere uger og til jeres slutprojekt, vil det oftest give mening
at skifte til framework Keras.  Emner

Kunstige neurale netværk - model og historie, træning af netværk (cost
function, optimeringsmetoder), grafisk illustration af netværk,
backpropagation, Keras / Tensorflow Litteratur

\i{§ 10 Introduction to Artificial Neural Networks with Keras} [HOML]

fra s. 289 til s. 307 (dvs. til og eksklusiv kapitel Building a Regression MLP Using the Sequential API).





L09: Deep learning og CNN's
Content

    Item
    Deep Learning: Convolutional Neural Networks
    Formål

    Vi slutter behandlingen af neurale net af med at kigge på 'Deep-learning' og de meget populære Convolutional Neural Networks; meget brugt til processering af billeder.

    Der kommer også en introduktion til brug af hardware (GPU'er og andet) i forbindelse med CNNs.
    Indhold
        Deep-learning,
        CNNs (Convolutional Neural Networks),
        Intro til GPU og andet hardware til Deep Learning.
    Litteratur

    \i{§ 14 Deep Computer Vision Using Convolutional Neural Networks} [HOML]

    Spring meget gerne over kapitler med TensorFlow kode og gennemgang af meget specifikke  Net's.

    Dvs.  skim eller spring over:

    \i{§  TensorFlow Implementation},  pp. 453-455 [HOML]

    \i{§  GoogLeNet , VGGNet, ResNet, "Xception, SENet, Implementing a
    ResNet-34 CNN Using Keras}, pp.  463-481 [HOML]

    \i{§ You Only Look Once (YOLO)} og resten af § 14, pp.489-496 [HOML].

    Opgaver
    Forberedelse inden lektionen

    Læs litteraturen.
    På klassen

    INGEN opgaver.
    Slides

     lesson09.pdf


 L10: Probabilistiske modeller
Content

    Item
    Probabilistiske modeller
    Attached Files:
        File L10_filer.zip Click for more options (976.494 KB)

    Formål :

    Sandsynlighedsteori danner et meget solidt fundament for machine learning og indgår i rigtigt mange sammenhænge. Selvom man godt kan bygge machine learning modeller uden særligt kendskab til sandsynlighedsregning, får I her lidt introduktion til emnet. Det vil sætte jer lidt bedre i stand til at forstå mange af de beskrivelser, som findes i fx. Scikit dokumentation og på nettet. Derudover er mange modeller direkte "probabilistiske modeller" - dvs. at de direkte modellerer data med funktioner fra sandsynlighedsteorien (såkaldte sandsynlighedsfordelinger).


    Emner :

    Introduktion til sandsynlighed i machine learning, probabilistiske classifiers, LDA/QDA, decision theory


    Litteratur :

    Følgende links kan være nyttige :

    https://en.wikipedia.org/wiki/Probabilistic_classification

    https://en.wikipedia.org/wiki/Naive_Bayes_classifier

    https://machinelearningmastery.com/how-to-calculate-joint-marginal-and-conditional-probability/

    https://machinelearningmastery.com/how-to-develop-an-intuition-for-probability-with-worked-examples/

    https://scikit-learn.org/stable/modules/lda_qda.html#lda-qda






 L11: Unsupervised I - PCA
Content

    Item
    Unsupervised learning 1 - PCA
    Attached Files:
        File L11_filer.zip Click for more options (2.577 MB)

    Formål :

    Vi skal især se på måder at få færre features / dimensioner i vores data. Det ønsker vi fx. for at undgå overfitting af en model eller for bedre at kunne visualisere data (i 2D/3D). Det kan også være for at komprimere data mængden. Vi skal især se på metoden Principal Component Analysis (PCA).

    Emner :

    Supervised vs. unsupervised learning, Curse of Dimensionality, dimensionsreduktion, manifold (mangfoldighed)

    Principal Component Analysis (PCA)

    Anvendelser – præprocessering, visualisering og kompression


    Litteratur :

    Kap. 8  s. 213- 225


 L12: Unsupervised II - Kmeans og GMM
Content

    Item
    Unsupervised learning 2 - Kmeans og GMM
    Attached Files:
        File L12_filer.zip Click for more options (840.417 KB)
    Yderligere i denne lektion (CEF):

    ITMAL AU-kursus-evaluering...husk at svare på evalueringen inden lektionsgangen.


    Formål :

    To hovedkategorier i machine learning er unsupervised og supervised learning. Regression og klassifikation er eksempler på supervised learning - dvs. vi kan træne en model da vi har både input data (feature værdier) og output data (= target data = labels for klasser). I nogle tilfælde har vi ikke output/target data - fx.  kunne en virksomhed opsamle data fra en masse brugere af deres udstyr og være interesserede i at vide om der er nogle mønstre i disse data, som fx. nogle "clusters" (sammenhængende områder/klumper) af brugere, men uden at vide det med sikkerhed. Denne form for machine learning kaldes unsupervised learning. Vi vil især se på de to meget almindelige metoder - Kmeans og Gaussian Mixture Models (GMM), som kan benyttes til clustering af data.


    Emner :

    Clustering, distance metrics, K-means og Gaussian Mixture Model algoritmerne

    Anvendelser - clustering, anomaly detection / outlier detection, density estimation


    Litteratur :

    Kapitel 9, om clustering - og især læs om K-means (s. 238-249).

    Optional : Gaussian Mixture Model - s. 260-269

    https://en.wikipedia.org/wiki/Cluster_analysis

    https://en.wikipedia.org/wiki/K-means_clustering

    https://scikit-learn.org/stable/modules/mixture.html

    https://scikit-learn.org/stable/modules/clustering.html#k-means


LXX: Poster-session
Content

    Item
    Poster-session
    Poster session aflyst, forår 2021!
    Sted

    Foregår på Zoom og Discord i ITMAL E20 gruppen. Først kursus-evaluering på Zoom (se link ovenfor) og derefter poster session på Discord - dvs. ren online undervisning.

    Poster-session foregår ved at Carsten og Peter "går rundt" til de forskellige grupper på Discord. Hver gruppe fortæller lidt om deres arbejde ud fra poster (uformelt). De andre studerende opfordres til at "følge med rundt" og høre om de andre gruppers projekter.
    Indhold
    i) Fælles kursusevaluering

    Kursus evaluering på 'klassen': 8:15 til ca. 8:45
    ii) Poster-session

    Poster-session på 'klassen'  fra ca. 8:45.

    (ca. 10 til 15 min per gruppe: Peter tager ITMAL Grp01+02+..., Carsten tager med ITMAL Grp30+29+...)

    Kriterier for poster - se menupunkt om Poster

    Husk at det er obligatorisk at deltage i poster-session.
    Forberedelse
        Upload en JPEG (eller lign) af jeres poster på Discorden, ITMAL, i jeres #Gruppe tekst kanal. Hav en eller flere personer i gruppen klar på at give en (uformel) mundtlig præsentation af posteren.
            Husk at jeres upload skal være i høj opløsning, f.eks. 2048 x 1048 eller højere.
            Discord viser et formindsket billede, men der er en "Åbn original" link nede i venstre hjørne, så man kan se billede i fuld opløsning!

























O4 mundtlig præsentation og evaluering
O4 mundtlig præsentation og evaluering

Hver gruppe skal vælge en dato, og i kan vælge i dato-listen nedenfor. I skal email'e til Carsten (cef@ase.au.dk) eller Peter  (pah@ase.au.dk) om jeres valg af dato, og vi tilføjer herefter til rækkefølge-listen (med et tilføjet tidspunkt).

Den mundtlige evaluering foregå på Zoom. Mød gerne op i god tid på Zoom: vi bruger breakout-room til evalueringen, så i kan snakke med de andre studerende 'på gangen udenfor breakout-room' inden i 'kommer ind'.

Hver gruppe vil 'komme ind' og skulle præsentere deres projekt. Det er kun underviser og gruppe som er til stede i 'lokalet' og det er her afgørende at ALLE deltagere i hver gruppe præsenterer deres individuelle bidrag til projektet.

Der er afsat i alt 15 minutter til hver gruppe, hvoraf jeres egen fremlæggelse forventes at være ca. 6-8 min. - herefter har underviser tid til at stille spørgsmål.

I skal ikke forberede en power point præsentation - I skal derimod hver især kort forklare om en del af projektet, som I har arbejdet med (fx. hvilken metode blev valgt, hvilke udfordringer var der, hvorfor blev denne metode valgt, noget om data analysen, etc.).

Efter hver gruppes præsentation og dialogen, vil hver enkelt gruppemedlem få at vide om kurset er "bestået/ikke bestået" samt feedback på O4 rapporten.

HUSK: Check at video udstyr/webcam og mikrofon virker - det er krav for at deltage.
Dato-listen

PAH:

    Fredag d. 21/5 - 8.00 - 12.00
    Tirsdag d. 25/5 - 12.00 - 16.00

CEF:

    torsdag d. 20/5   (tidsrum 08:30 til 11:30) FYLDT OP
    fredag d. 21/5      (tidsrum 10:00 til 13:00)
    tirsdag d. 25/5    (tidsrum 08:30 til 16:00)
    onsdag d. 26/5    (tidsrum 08:30 til 14:00)
    torsdag d. 27/5   (tidsrum 08:30 til 16:00)

Rækkefølgeliste

PAH :

(ZOOM LINK for begge PAH dage :  https://aarhusuniversity.zoom.us/j/69161795618)

Fredag d. 21/5 :

    Kl. 8:00 - Grp 29
    Kl. 8:15 - Grp 24
    Kl. 8:30 - Grp 31
    Kl. 8:45 - Grp 27
    Kl. 9:00 - Grp 13
    Kl. 9:15 - Grp 2
    Kl. 9:30 - Grp 34
    Kl. 10:00 - Grp 7

Tirsdag d. 25/5 :

    Kl. 12:00 - Grp 4
    Kl. 12:15 - Grp 33
    Kl. 12:30 - Grp 15
    Kl. 12:45 - Grp 30
    Kl. 13:00 - Grp 6
    Kl. 13:15 - Grp 16
    Kl. 13:30 - Grp 41
    Kl. 13:45 - Grp 12
    Kl. 14:00 - Grp 3
    Kl. 14:15 - Grp 20
    Kl. 14:30 - Grp 35

(CEF: se nedenfor)
