COURSE
LESSON L00

\h4{Forberedelse inden kursusstart}

\h5{Formål}

\p{Gruppe tilmelding: tilmeld dig til en ITMAL gruppe (se Blackboard link
ovenfor).}

\p{Installation: de obligatoriske værktøjer til ITMAL inden kursusstart (dvs.
L01).}

\p{Forberedelse til L01: Hent GIT repositories til litteraturen [GITHOML], prøv at
kører et par Jupyter Notebooks [JPYNB], og læs mere om pythons NumPy
bibliotek.}

\p{Ekstra materiale til forberedelse: optionelle python opgaver, hvis du vil sætte
dig mere ind i sproget.}

\h5{Installation}

\ul{
	\li{Installer Anaconda på din PC:}
	\ul{
	\li{\link{www.anaconda.com/products/individual,https://www.anaconda.com/products/individual}}
		\li{vælg 'Download' (downloader direkte for Windows),}
		\li{eller vælg Linux eller Mac, 32 eller 64 bit (dit valg),} 
		\li{nværende nyeste Anaconda3 version er \bf{2021.05}}
	}
	\ul{
		\li{ALTERNATIV 1:}
		\ul{
			\li{brug vores ASE GPU Cluster som jupyter hub server,}
			\li{se info in Kursusinfo | GPU Cluster.}
		}
		\li{ALTERNATIV 2:}
		\ul{
			\li{Lav en konto på Google's Colaboratory,}
			\li{\link{colab.research.google.co, https://colab.research.google.com}}
		}
	}
	\li{Test at du kan køre jupyter notebooks [JYPYNB] fra [GITHOML], prøv f.eks.:}
	\ul{
		\li{index.ipynb}
	}
}

\h5{Forberedelse til Lektion 01}

\ul{
	\li{Læs kursusinfo via Blackboard-linket ovenfor.
	\li{Få fat i litteratur til kurset.
	\li{Clone [GITHOML] til din egen PC, se how-to under kursusinfo | kursusforkortelser.
	\li{Skim disse tutorials igennem:}
	\ul{
		\li{§ Scientific Python tutorials: NumPy}
		\li{tools_numpy.ipynb [GITHOML]}
}

\p{Læs blot, hvad du finder relevant så som 'iteration', men spring blot over
emner, der er for komplekse eller for 'pythoniske', så som 'Stacking arrays' og
'QR decomposition'.  Note vdr.  kildekritik og 'informations-overload'}

\p{Vi vil i dette kurset tit kunne blive overvældet af for meget ekstern
information (informations-overload), så du skal danne dig en metode til at
kunne selektere og navigere i materialet.}

\p{Vi vil primært holde os til [HOML], [GITHOML] og Scikit-learn, med en note
om, at nettet flyder over med ekstra (til tider ubrugelig/ufiltreret)
information: en kildekritiks holdning er vigtig!  Ekstra materiale til
forberedelse}

Hvis du har brug for at opfriske dit lineær algebra matematik eller er helt ny
til python, så kan du f.eks.  læse/skimme følgende notebooks, i prioriteret
rækkefølge:

	   [OPTIONAL] Python og vectors/matrices math: math_linear_algebra.ipynb [GITHOML]
	   [OPTIONAL] Python og grafisk plotting: tools_matplotlib.ipynb [GITHOML]
	   [OPTIONAL] Ekstra, Python og dataværktøjet 'Pandas' : tools_pandas.ipynb [GITHOML]
	   [OPTIONAL] Ekstra, mest for de matematik intereserede:  math_differential_calculus.ipynb [GITHOML]

\p{Pandas er et meget populært databehandlingsværktøj, men vi kommer ikke til at
bruge Pandas i dette kursus.}

LESSON L01

\h1{Introduktion}

\h2{Formål}

Denne lektion har til formål at give indledende information om kurset.  Dvs. 
at vi præsentere de formelle rammer vdr.

\ul{
	   \li{ITMAL gruppetilmelding,}
	   \li{opgavesæt og journalafleveringer,}
	   \li{eksamensform,}
	   \li{Blackboard opbygning og fildeling.}
}

\p{Herefter vil vi præsentere machine learning [ML] som koncept overordnet, og
kort ridse lektionsplanen for kurset op.}


\p{Software til brug for kurset introduceres og skal installeres på jeres PC'er,
se 'L00: Forberedelse' for en installationsguide.  Vi anvender python
distributionen anaconda og i henter og installere den sidste nye version.  På
klassen vil der blive givet en kort demo af jupyter notebooks, dvs.  et at de
udviklingsværktøjer til python vi vil bruge.}

\p{Vi kigge på Scikit-learn, det primære eksterne web-sted vi vil bruge i kurset,
samt forsøge os med et par små programmer i python.}

\p{Til slut kigger vi på supervised learning og at kunne predicte
'life-satisfactory' via demo projektet i [HOML], og vi ser på pythons modul- og
klassebegreber (modules, classes), så vi kan genbruge kode i senere
lektioner..}

\h2{Indhold}

\ul{
	   \li{Formelle rammer vdr. kurset.}
	   \li{Eksamensform, godkendelsesfag via:}
	   \ul{
	      	\li{et sæt obligatoriske skriftlige gruppe-journaler med afleveringsdeadlines,}
	      	\li{en poster-session, med aflevering af poster og mundtlig præsentation af poster,}
	      	\li{en mundtlig gennemgang af den sidste journal med alle medlemmer i ITMAL gruppen, samt evaluering af hver gruppemedlems bidrag.}
		\li{=> Endelig godkendelse af kurset sker på en samlet vurdering af de tre punkter ovenfor.}
	}
	   \li{Læringsmål.}
	   \li{Litteratur.}
	   \li{Intro til software, der bruges i ITMAL:}
	   \ul{
	      	\li{python generelt (link til mini python intro: demo.ipynb Click for more options)}
		\li{anaconda python distribution:}
		\ul{
	          	\li{jupyter notebooks,}
	          	\li{spyder developer environment.}
		}
		\li{Scikit-learn,}
		\li{opgave med python modul og klasser.}
	}
	   \li{Intro til machine learning:}
	   \ul{
	       \li{Supervised learning (regression): 'life-satisfactory' [HOML].}
	}
}

\h2{Litteratur}

	\px{§ Preface, p. xv [HOML] (eksklusiv fra Using Code Examples...og resten af intro	kapitlet)}

	\px{§ 1 The machine Learning Landscape [HOML]}

	\px{§ 2 End-to-End Machine Learning Project [HOML]}

\p{Dette kapitel indeholder mange nye koncepter og en del kode.  Vi vender
senere tilbage til kapitlet senere, så læs det og prøv at danne dig et overblik
(dvs.  nærlæs ikke).}

\p{Når du har installeret anaconda (se L00):}

	\px{§ Scientific Python tutorials: NumPy}
	
	\px{tools_numpy.ipynb [GITHOML]}

\p{Læs blot, hvad du finder relevant så som 'iteration', men spring blot over
emner, der er for komplekse eller for 'pythoniske', så som 'Stacking arrays' og
'QR decomposition'.}

\h2{Opgaver}

\p{Forberedelse inden lektionen}

\ul{
	\li{Meld dig ind i en ITMAL working-group [G].}
	\li{Følg installation processen givet i lektion nul ('L00: Forberedelse').}
	\li{Læs pensum.}
}

\h5{På klassen}

\ul{
	\li{Diskussion om ML (indlejret i forelæsningen).}
	\li{Opgave: intro.ipynb}
	\li{HUSK DATA til intro'en (download og udpak så "dataset" dir ligger sammen med intro.ipynb): datasets.zip Click for more options}
	\li{Opgave: modules_and_classes.ipynb}
}

\h5{Optionelle opgaver}

\p{Se 'Ekstra materiale til forberedelse' i lektion 'nul', specielt hvis du har brug for en python og lineær algebra
kick-start.}

\h5{Slides}

\px{
	\link{,[GITMAL]/L01/lesson01.pdf}
}

REFS 
	[ITU]    https://itundervisning.ase.au.dk/ITMAL_E21/Html
	[GITMAL] https://itundervisning.ase.au.dk/ITMAL_E21
END


<!DOCTYPE html>
<html>
<head>
</head>
<body style="font-family: Lato;font-size: 12pt;color: #494c4e;">
<h5>Form&aring;l</h5>
<p>Opsamlingslektion: vi tager et genblik p&aring; <em>&sect; 2 <em>"End-to-End Machine Learning Project",</em></em> og samler op p&aring; dette brede kapitel.</p>
<p>Vi g&aring;r f&oslash;rst igang med at gennemg&aring; <em><strong>K-fold Cross-validation</strong></em> (eller K-fold CV), for derefter at bruge "The Map" til at komme igennem alle grundliggende koncepter i <em>&sect; 2. </em><em><br /></em></p>
<p>Da alle kerne-koncepter i supervised ML nu kendes, kan det hele konkret sammens&aelig;ttes i en samlet processerings-<em><strong>pipeline</strong>.</em> Programmerings-teknisk ser vi derfor til sidst p&aring; Scikit-learns Pipelines.</p>
<ul>
<li><a rel="noopener" href="https://blackboard.au.dk/bbcswebdav/pid-2931054-dt-content-rid-10655397_1/xid-10655397_1" target="_blank" title="ml_supervised_map.pdf">Oversigtskortet for Supervised learning</a>:</li>
</ul>
<p style="margin-left: 60px;"><a rel="noopener" href="https://blackboard.au.dk/bbcswebdav/pid-2931054-dt-content-rid-10655397_1/xid-10655397_1" target="_blank" title="ml_supervised_map.pdf"><img src="https://blackboard.au.dk/bbcswebdav/pid-2931054-dt-content-rid-10655394_1/xid-10655394_1" alt="ml_supervised_map.pdf" title="ml_supervised_map.pdf" width="512" height="499" /></a></p>
<h5>Indhold</h5>
<ul>
<li>Generel genl&aelig;sning og repetition af <em>&sect; 2</em></li>
<li>K-fold Cross-validation</li>
<li>Pipelines<em></em></li>
</ul>
<h5>Litteratur</h5>
<p style="margin-left: 30px;"><em>Genl&aelig;s: &sect; 2 <em>"End-to-End Machine Learning Project" </em></em>[HOML]</p>
<p style="margin-left: 60px;">(eksklusiv <em>"Create the Workspace" </em>og <em>"Download the Data")</em></p>
<p style="margin-left: 30px;"><span style="color: #800000;"><span style="font-family: courier new, courier; color: #000000;"><a rel="noopener" href="https://scikit-learn.org/stable/modules/generated/sklearn.model_selection.KFold.html?highlight=k%20fold#sklearn.model_selection.KFold" target="_blank"><span style="color: #000000;">Scikit's dokumentations-side vdr. k-fold CV</span></a></span><em> </em></span></p>
<h6>Forberedelse inden lektionen</h6>
<ul>
<li>L&aelig;s litteraturen.<em><br /></em></li>
</ul>
<h6>P&aring; klassen</h6>
<ol>
<li><span style="text-decoration: line-through;">'</span>Sp&oslash;rge-minutter'</li>
<li>Almindelig forel&aelig;sning
<ul>
<li>ekstra materiale: <span style="font-family: courier new, courier;"><a rel="noopener" href="https://blackboard.au.dk/bbcswebdav/pid-2931054-dt-content-rid-10655399_1/xid-10655399_1" target="_blank">k-fold_demo.ipynb</a></span></li>
</ul>
</li>
<li>Exercise: <span style="font-family: courier new, courier;"><a rel="noopener" href="https://blackboard.au.dk/bbcswebdav/pid-2931054-dt-content-rid-10655389_1/xid-10655389_1" target="_blank">pipelines.ipynb</a> <br /></span>
<ul>
<li><span style="color: #800000;"><span style="color: #000000;">Data til pipelines opgaven (b&oslash;r lige i L07/Data/</span>)</span>: <span style="font-family: courier new, courier;"><a rel="noopener" href="https://blackboard.au.dk/bbcswebdav/pid-2931054-dt-content-rid-10656401_1/xid-10656401_1" class="event_clickFileName" target="_blank">itmal_l01_data.pkl</a> </span></li>
<li>(Du har allerede denne data-fil, hvis du pull'er fra GITMAL)</li>
</ul>
</li>
</ol>
<h5>Slides</h5>
<p style="margin-left: 30px;"><span style="font-family: courier new, courier;"><a rel="noopener" href="https://blackboard.au.dk/bbcswebdav/pid-2931054-dt-content-rid-10655390_1/xid-10655390_1" target="_blank">lesson03.pdf</a></span></p>
</body>
</html>